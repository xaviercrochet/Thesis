\chapter{Conclusion}
\label{conclusion}
\lhead{Chapitre 6 - Conclusion}
Une solution, sous forme de fonctionnalité, a été apportée pour chacun des objectifs mis en avant dans les prémices de ce mémoire. Cependant, il faut garder à l'esprit que la solution présentée dans ce mémoire est le résultat d'une première itération de développement. Il serait insensé de penser que l'application et chacune de ses fonctionnalités se trouve dans un état de plénitude, la section \ref{futur_work} en témoigne. 

Néanmoins, l'objectif principal de ce mémoire, à savoir - \textit{proposer une solution qui automatise la gestion des programmes de cours afin de soulager les commissions de programmes de la charge de travail que constitue la création, maintenance et vérification des programmes de cours} - a été rempli avec succès.

Il est maintenant possible, du coté de la commission de programme, de créer, maintenir et faire évoluer des programmes de cours de façon intuitive et efficace. Le processus, qui amène la commission à vérifier les programmes des étudiants est dorénavant plus simple et surtout nettement moins couteux en temps pour les deux parties.

Enfin, du coté des étudiants, la configuration et la gestion de leurs programmes de cours est plus limpide. Ayant désormais une meilleure connaissance des tenants et aboutissants des contraintes qui régissent les programmes de cours, il est plus aisé pour ces derniers de fournir chaque année à la commission,  des programmes qui tiennent mieux la route. 

Pour finir, je suis satisfait, dans le cadre de ce mémoire, d'avoir pu présenter un projet dont les exigences, compte tenu des personnes à qui l'application s'adresse, sont assez élevées. J'espère avoir la chance, dans ma vie professionnelle, de pouvoir travailler sur des projets qui susciteront mon intérêt autant que l'a fait ce mémoire.

