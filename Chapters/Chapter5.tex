\chapter{Validation}
\label{validation}
\lhead{Chapitre 4 - Validation}
\section{Introduction}
\label{introduction_val}
Ce document présente un scénario typique d'utilisation pour la \textbf{commission INFO} ainsi qu'un scénario d'utilisation pour un \textit{Étudiant}. L'application au moment où ce scénario a été conçu n'est pas encore finie. Il risque d'y avoir des changements au niveau de son design et l'ajout de certaines \textit{features} non encore implémentées. Cependant, la base de l'application est suffisamment présente que pour permettre à ce test d’être pertinent.

Le scénario \textit{Commission INFO} consiste à créer un catalogue de cours, mettre à jours ses informations et créer un programme de cours à la carte. 

Le scénario \textit{Étudiant} consiste à se créer un compte, se connecter avec sur l'application, créer un programme de cours et à le configurer 

Notez que 

1 - Le catalogue présenté aux étudiants pour construire leur programme de cours est le dernier en date à voir été créé en base de donnée (la feature pour sélectionner le programme de cours principal n'ayant pas encore été implémentée)

2 - Si la commission ne rajoute pas l'information relative au semestre durant lequel sont dispensés les cours, l'étudiant ne verra aucun cours lorsqu'il voudra créer une année (La propriété \textit{Semester} étant initialisée à \textit{NONE})
\label{note2}

3 - L'étudiant ne peut pas accéder aux informations relatives au programme qu'il veut suivre depuis son interface (Ces vues n'ont pas encore été implémentées)

Il est demandé de ne pas regarder dans le manuel pour réaliser l'expérience, afin que le feedback soit le plus complet possible.

Bon amusement :-)

\section{Ressources}
Tout d’abord, voici l'url de l'application : : \url{http://curriculum-mgmt.herokuapp.com/} 

Ensuite, voici les informations relatives au compte avec lequel il faut se connecter pour accéder à l'application
\begin{itemize}
  \item USERNAME: commission@gmail.com
  \item PASSWORD: coucou42
\end{itemize}

(Il n'est pas possible de se créer un compte admin via l'application pour des raisons de sécurité évidentes)

\section{Scénario Commission INFO}

\begin{enumerate}
  \item Connectez vous à l'application
  \item Accédez à l'onglet \textbf{Catalogues}
  \item Créer un graphe avec yEd, exporter le graph en .graphml ou utiliser un fichier de graphe déjà existant
  \item Créer un catalogue en utilisant le fichier de graphe précédemment créer
  \item Mettre à jour les informations du catalogue (En commençant par télécharger le fichier excel depuis l'application, comme expliqué sur la vue)
  \item Se rendre dans l'onglet \textbf{Programmes} pour accéder aux programmes de cours
  \item Créer un nouveau programme de cours avec les modules \& cours désirés (Attention, tout les cours des modules sélectionnés seront ajoutés automatiquement, vous ne pouvez que choisir les cours qui ne sont dans aucun modules)
  \item Supprimer le programme de cours précédemment créer
  \item Naviguer dans les différents menus

\end{enumerate}

\section{Scénario Étudiant}


\begin{itemize}

  \item Créer un comte sur l'application. Vous pouvez mettre n'importe quelle adresse email, aucun mails ne sera envoyé. 
  \item Se rendre dans le menu à droite, cliquez sur mon compte et changer votre mot de passe. Vous pouvez aussi supprimer votre compte si vous le désirer
  \item Se rendre dans le menu \textit{Mes programmes de cours} et se créer un nouveau programme
  \item Configurer son programme de cours, en choisissant des modules par exemple, et en ajoutant une année avec les cours désirés. Aucun cours ne sera afficher si la note 2 \ref{note2} de la section \ref{introduction_val} n'a pas été suivie. 
  \item Vérifier les contraintes de son programme
  \item Envoyer son programme à la validation
  \item Naviguer dans les différents menus

\end{itemize}