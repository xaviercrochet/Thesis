\chapter{Déploiement de l'application}
\lhead{Annexe 1 - Déploiement}
\section{Déployer l'application sur Heroku}
Tout d'abord, il faut aller chercher le code source de l'application sur le répertoire github (se référer à l'annexe \ref{ressources} pour le lien)

Ensuite, il faut créer une application sur la plateforme Heroku. 

Après, pour déployer une application existante sur Heroku,  il suffit d'ajouter le lien vers le \textit{remote Heroku} dans le fichier de configuration git:

\begin{lstlisting}
git remote add heroku git@heroku.com:{nom_de_l_application_heroku}.git
git push heroku master
\end{lstlisting}

Pour plus de détails, se référer à la documentation officielle \cite{Heroku}.

\section{Configurer l'application pour qu'elle communique avec le cloud Amazon}
Les formulaires Excel et les fichiers de graphes sont hébergés sur le cloud Amazon (le répertoire étant en lecture seule). 

Tout d'abord, il faut se créer un \textit{bucket} sur le cloud Amazon qui va contenir tout les fichiers qui y seront téléchargés depuis l'application. Il suffit de se rendre sur la page \url{http://aws.amazon.com/s3/}, de s'y créer un compte et de créer un bucket.

Pour configurer l'application afin qu'elle télécharge les fichiers vers ce \textit{bucket}. Il faut configurer l’environnement de production comme suit:

\begin{lstlisting}
# config/environments/production.rb
config.paperclip_defaults = {
  :storage => :s3,
  :s3_credentials => {
    :bucket => ENV['S3_BUCKET_NAME'],
    :access_key_id => ENV['AWS_ACCESS_KEY_ID'],
    :secret_access_key => ENV['AWS_SECRET_ACCESS_KEY']
  }
}
\end{lstlisting}

Les deux dernières variables étant disponibles depuis la console de configuration du cloud amazon s3 \cite{AmazonS3}.

Ensuite, il faut ajouter ces variables dans l'application Heroku:

\begin{lstlisting}
$ heroku config:set S3_BUCKET_NAME=your_bucket_name
$ heroku config:set AWS_ACCESS_KEY_ID=your_access_key_id
$ heroku config:set AWS_SECRET_ACCESS_KEY=your_secret_access_key
\end{lstlisting}

Pour plus de détails, se référer à la documentation officielle \cite{AmazonS3}.