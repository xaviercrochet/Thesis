\chapter{Déploiement de l'application}
\lhead{Annexe 1 - Déploiement}
\section{Déployer l'application sur Heroku}
Tout d'abord, il faut aller chercher le code source de l'application sur le répertorie github (se référer à l'annexe \ref{ressources} pour le lien)

Ensuite, il faut créer une application sur Heroku comme expliqué dans la documentation disponible ici \cite{Heroku}

Après, pour déployer une application existante sur Heroku il suffit d'ajouter le lient vers le remote Heroku dans le fichier de configuration git:

\begin{lstlisting}
git remote add heroku git@heroku.com:{nom_de_l_application_heroku}.git
git push heroku master
\end{lstlisting}

\section{Configurer l'application pour qu'elle communique avec le cloud Amazon}
Les formulaires excels et les fichiers de graphes sont hébergés sur le cloud Amazon (le répertoire étant en lecture seule). 

Tout d'abord, il faut se créer un \textit{bucket} sur le cloud Amazon qui va contenir tout les fichiers qui y seront téléchargés depuis l'application. Il suffit de se rendre sur la page \url{http://aws.amazon.com/s3/}, de s'y créer un compte et de créer un bucket.

Pour configurer l'application afin qu'elle télécharge les fichiers vers ce \textit{bucket}. La documentation \cite{AmazonS3} explique en détails comment configurer l'application afin qu'elle communique correctement avec le bucket. 